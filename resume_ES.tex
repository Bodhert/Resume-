% !TeX program = xelatex
% Run with XeLaTeX

\documentclass[
    changecolor={111, 156, 45}, 
%    titlecolor=second,
%    colorharmony={wheel,1,3}, 
]{cv-roald}
% Available options for cv-roald documentclass: 
% 
% - changecolor = {R, G, B} (default: {169, 53, 40} = sharelatex lion red)
%   Changes the color of the header bar and the titles. 
%   It expects a RGB color code where RGB go from 0-255.
%
% - colorharmony = {wheel,i,j} (defualt: {wheel,1,3})
%   Choose the color harmony model you want to use.
%     wheel = wheel or twheel
%     i		= number < j
%	  j		= for j-color harmony 
%	    	  j=3 -> three color harmony or color triad
% 	See page 37 of the xcolor manual for more information on commands: 
% 	http://texdoc.net/texmf-dist/doc/latex/xcolor/xcolor.pdf  
% 	Or check http://paletton.com to see how color harmonies work.
%
% - titlecolor = {colorname} (default: main)
%	Change the colors of the titles. You can try: second, black!50, 
%	green!40!yellow, etc.     	
%
%  The colors defined through changecolor and colorharmony are called 
%  *main* and *second* and can be used in the main document.	
\newcommand{\RomanNumeralCaps}[1]
    {\MakeUppercase{\romannumeral #1}}

\begin{document}
\pagestyle{empty} %to remove the page numbers

% This is the header on the first page. It contains your name and contact
% details. 
% \sep inserts a | between items. 
% You can use FontAwesome icons and use \FAspace after a font awesome icon to
% insert a predefined horizontal space after a font awesome icon icon.
\header{Alejandro}{Cordoba}
 {%
  \faMapMarker \FAspace Medellín \sep Colombia \sep%
  \faMobile \FAspace +57 301 323 8497\sep% 
  \faGithub Bodhert \FAspace \link{https://github.com/bodhert}
 }
 {%
  \faEnvelope \FAspace alejandro.cordoba.bodhert@gmail.com \sep% 
  \faLinkedinSquare  \FAspace Alejandro Cordoba Bodhert \FAspace \link{https://bo.linkedin.com/in/alejandro-cordoba-bodhert-746598123}
 }


% Add a picture to the top right of the page. Comment or delete if you do not 
% want a picture


\textit{"La disciplina tarde que temprano vencerá la inteligencia" , \textit{Yokoi Kenji}.}

\section*{Perfil}

 Amante de la tecnología y de resolver problemas con la misma, Entiendo como es su funcionamiento y acoplamiento. Me adapto fácil a entornos que están en constante cambio, con bases solidas en diseño y funcionamiento de algoritmos y software. La \textbf{comunicación} y el \textbf{trabajo en equipo} son mis mejores habilidades blandas, teniendo en cuenta siempre el \textbf{respeto}. \textbf{Disciplinado}, \textbf{motivado} por mis metas personales y por las metas grupales. Me gusta mucho aprender de las personas y aprender con ellas.

\section*{Experiencia}

\begin{tabularcv}
2022-2024 & \worktitle{Desarrollador FullStack MONO (Fintech)}
\newline
\newline
En el dinámico entorno de una \textbf{Startup}, participé en diversas actividades clave para respaldar el negocio y la operación transversalmente. Mi enfoque principal fue el desarrollo, mejora e incorporación de nuevas características al producto de software. Diseñé, planifiqué, creé y ejecuté \textbf{pruebas de conceptos} para identificar potenciales clientes y oportunidades en el mercado (\textit{Product Market Fit}). Además, realicé inducciones y entrenamientos para nuevos desarrolladores, proporcioné soporte técnico de la aplicación en producción y atención al cliente final. También capacité a otras áreas como negocio y diseño para asegurar una comprensión integral del producto y alineación de objetivos. Todas estas actividades se realizaron en torno a \textbf{Elixir} y su ecosistema, utilizando \textbf{Phoenix} como web framework, \textbf{LiveView}, \textbf{JavaScript}, \textbf{Tailwind} para el frontend, y \textbf{Postgres} como base de datos. Para nuestro ecosistema de desarrollo, nos basábamos fuertemente en \textbf{Docker}, lo que nos permitía preparar nuestros ambientes locales y agilizar significativamente el proceso de desarrollo y despliegue. Esta combinación en particular se eligió para garantizar alta disponibilidad, crucial en el sistema financiero, velocidad de desarrollo y facilidad al trabajar con sistemas concurrentes, sin que escalar resultara en mayores costos operativos y monetarios. Se organizó el equipo utilizando diversas \textbf{metodologías ágiles}. Adoptamos elementos de \textbf{Scrum}, como planning poker, dailies y retros. También implementamos Kanban para visualizar el flujo de trabajo. Se ensayó el marco de \textbf{Shape Up}, el cual genera dinámicas diferentes para la ejecución de tareas y proyectos.


    \begin{itemize}
          \item Crear y mantener características del código existente, enfocándome en proporcionar el máximo valor a los clientes.
    \item Analizar y solucionar errores para asegurar la calidad del producto.
    \item Proponer e implementar mejoras al código existente mediante revisiones de código, optimización de procesos internos y creación de nuevas características, agilizando la entrega de valor.
    \item Desarrollar un producto mínimo viable: una \textbf{aplicación móvil} billetera virtual de marca blanca, utilizando nuestra \textbf{API}. Usé herramientas de \textbf{low code} como \textbf{FlutterFlow} para el frontend, y \textbf{Python} junto con \textbf{FastAPI} para el backend. \textbf{Supabase} se utilizó para el almacenamiento de datos y como proveedor de autenticación.
    \item Leer código \textbf{open source} para adaptarlo a las necesidades internas.
    \item Participar en soporte técnico rotativo, utilizando herramientas de monitoreo como \textbf{AppSignal} para identificar y solucionar fallos críticos mediante revisión de código, \textbf{SQL} y análisis de logs.
    \item Asumir responsabilidad y autoridad desde el principio hasta el fin del \textit{feature} entregado.
    \item Implementar automatizaciones internas utilizando herramientas \textbf{low code} como \textbf{n8n}, según las necesidades del negocio.
    \item Diseñar y planificar desarrollos importantes utilizando \textbf{PlantUML} y \textbf{Excalidraw} para generar diagramas, fomentar la discusión, anticipar fallos y revisar la factibilidad y alcance de los requerimientos.
    \item Colaborar con el equipo de \textbf{UX} para optimizar la experiencia del usuario en el producto.
    \item Colaborar con operaciones para optimizar la infraestructura y mejorar los procesos de CI/CD.
    \end{itemize}
\end{tabularcv}

\begin{tabularcv}
2021-2022 & \worktitle{Desarrollador FullStack Freelance}
\newline
En cargo de diversas actividades, que principalmente incluían desarrollar en \textbf{Elixir}, usando \textbf{Phoenix} como framework y \textbf{Liveview} o \textbf{Javascript} como Front end.

\begin{itemize}
    \item Crear y probar funciones relacionadas con la parte administrativa de una landing page.
    \item Leer y modificar código libre de una plataforma social, escrita en Elixir (\textbf{Pleroma}) para presentar una prueba de concepto, la cual censuraba publicaciones si el contenido era inadecuado. sin la necesidad de moderadores humanos.
    \item \textbf{WEB3}, \textbf{Blockchain}, especialmente en las cadenas de \textbf{Ethereum(Solidity)} y \textbf{Solana (Rust)}, en la cual se realizó diferentes pruebas de conceptos y analizó la factibilidad de crear e implementar diferentes protocolos en las diferentes cadenas, junto con sus ventajas y desventajas, para la comunicación se utilizó los clientes disponibles de las respectivas cadenas en \textbf{Javascript}.
\end{itemize}
\end{tabularcv}

\begin{tabularcv}
2019-2021   &   \worktitle{Ingeniero Devops}{Perficient/PSL (Productora de software)}
                \newline Encargado de administrar, mejorar y dar soporte al ciclo de vida del desarrollo de software (CI/CD), trabajando conjuntamente con QA, negocio, desarrollo para una entrega rápida, continua, segura y de calidad de los productos.
                 \begin{itemize}
                  \item Uso de herramientas para la gestión de la configuración como \textbf{Puppet} y \textbf{Ansible}
                  \item Administración de herramientas cloud en \textbf{AWS} como: \textbf{VPC(EC2 machines)}, \textbf{RDS}, \textbf{Elastic Search Service}, \textbf{Glue}, Acceso y restricción de los mismos servicios mediante \textbf{Policies}.
                  \item Automatización para la creacion de infraestructura mediante \textbf{Terraform}
                  \item Automatización de tareas mediante scripts en \textbf{python} y \textbf{sh}
                  \item Administración de código en \textbf{git} y \textbf{svn} 
                  \item Creación y mantenimiento de pipelines y jobs para facilitar procesos de \textbf{(CI/CD)} en \textbf{Jenkins}, usando pipelines como código  \textbf{Groovy}, \textbf{JJB} o \textbf{sh, bash}
                \end{itemize} 
\end{tabularcv}

\begin{tabularcv}
2017-2018   &   \worktitle{Desarrollador FullStack}{Universidad EAFIT (Departamento de ingeniería de producción)}
                \newline Aplicación para la optimización de  consumo de energía en carros eléctricos.
                \begin{itemize}
                % Example use of *maincolor* and *secondcolor* in the main document.    
                  \item Programación en \textbf{Java} de servicios backend \textbf{(Spring)}, con herramientas de codigo libre como \textbf{osrm} y \textbf{postgreSQL}. Trabajando con datos reales que alimentan modelos de decisión. 
                  \item Administración y despliegue en servidores \textbf{Debian}.
                  \item Creación de servicios \textbf{REST} para ser consumidos por aplicaciones.
                  \item Automatización de procesos en Java
                \end{itemize} 
\end{tabularcv} 



\section*{Educación}
% Use tabularcv environment to make a two column environment. The left column 
% is for the dates, the right one is for details of your education for example. 
% You can use the command \worktitle{Study name/Job title}{Location}.
\begin{tabularcv}

2015-2019   &   \worktitle{Ingeniería de sistemas}{Universidad EAFIT} \link{https://drive.google.com/file/d/1qoOgISk8u03fq9ACYFR2u8Se5Dq4TEuw/view?usp=sharing}
                \newline Pasión por la  \textbf{programación}, \textbf{la algoritmia}, \textbf{automatización} de procesos y la administración de sistemas basados en \textbf{unix}, Además de poseer conocimiento en diferentes metodologías de desarrollo, tanto tradicionales como ágiles,  con fortaleza en \textbf{scrum}, consciente de las buenas practicas, patrones y culturas de desarrollo como \textbf{DevOps} y \textbf{código limpio}.
                \begin{itemize}
				  \item Realización de diversos proyectos que cubren tópicos como programación para sistemas operativos \textbf{(POSIX)}, \textbf{arduino}, \textbf{programación móvil(Ionic, Android)},  \textbf{web services(AWS)}, \textbf{programación funcional}, \textbf{data science}.\link{https://drive.google.com/open?id=1DdJ-zSWe93dtckBD8_2KfLKlCP9mtG4L}.
				  \item Integrante del semillero de \textbf{programación competitiva} desde el 2016.
				  \item  Habilidad en la solución de problemas por medio de la \textbf{algoritmia}.
				  \item Motivado por el auto-aprendizaje.  
                \end{itemize}
                \\
                
2003-2014   &   \worktitle{Bachiller}{Colegio San José de la Salle}  \link{https://drive.google.com/open?id=1Z0-YVHVlFtDL2GwGxhAoMrZUSDcweuF7}
                \newline Lasallista, Educado en un ambiente que promueve una formación integral en todos los aspectos humanos, promoviendo una sana convivencia. Destaque mi \textbf{liderazgo} llegando a ser representante de los estudiantes y jefe de grupo, en mí ultimo año. \link{https://drive.google.com/open?id=160q1YruZjghGWJOQEOUo_zYKsZ1OVReW}
                \begin{itemize}
				  \item Realización de un proyecto \textbf{investigativo} enfocado al comportamiento social de los estudiantes y como afecta esto su desempeño académico.
                  \item \textbf{Orador} en \RomanNumeralCaps{2} foro de ciencias sociales.
                \end{itemize} 

                
\end{tabularcv}


\section*{Logros}
% Use \link{URL} to place an link. It inserts a fontawsome logo, where the color is determined from the color harmony settings.
\begin{tabularcv}	
2017    &   Mención honorable en maratón Sur Americana (región norte) ACM ICPC. \link{https://drive.google.com/open?id=1GoiMosLKJv-FpBzl04NuQETugrnbYnaB}\\
2017	&	Puesto 32 en maratón Nacional de programación ACM ICPC.  \link{https://drive.google.com/open?id=1dNH9uwUQMZJxrwjlYxjF_f4irCCxR1Gn}\\
2015    &   Puesto 41 en maratón Nacional de programación ACM ICPC  . \link{https://drive.google.com/file/d/1Zkp_3n9bve-QKLGidpJzQSrKxTxuB_bi/view?usp=sharing} \\

                
\end{tabularcv}

\section*{Habilidades tecnológicas}
    \textit{Provedores Cloud:} \textbf{AWS, IBM, AZURE.}\\
    \textit{Configuración de servidores:} \textbf{NGINX, Apache.}\\
    \textit{Herramientas de construcción del codigo:}  \textbf{MIX, PIP, npm, cmake, MVN.}\\
	\textit{Lenguajes de programación:}  \textbf{Elixir, Python, Javascript, SQL, C++, Java, Solidity, Visual Basic For Excel, C\#}\\
	\textit{CI/CD:} \textbf{Github actions, Jenkins, Unit and Integration testing, pytest, Junit Framework.}\\
    \textit{Monitoreo:} \textbf{AppSignal, Sentry, Splunk, Kibana}\\
	\textit{Bases de Datos:} \textbf{PostgreSQl, RDS(AWS),  MongoDB, Oracle, MySql.}\\ 
	\textit{Sistemas operativos:} \textbf{MacOs, Sistemas basados en linux(principalmente de la familia de debian), Windows.}\\
	\textit{Administración de código: } \textbf{Git, Github, Gitlab, Bitbucket y Svn.}\\
	\textit{Herramientas:} \textbf{linux shell, zsh.}\\
	\textit{Frameworks:} \textbf{LiveView, Phoenix, FastAPI, Spring, node, rails, Ionic, Android Studio, FlutterFlow}. \\
	\textit{Virtualización:} \textbf{Docker, Virtual Box, Vagrant, K8s}.\\
	\textit{Hardware:} \textbf{Raspberry pi, Arduino}.\\
	\textit{Otras herramientas:} \textbf{Microsoft office(especialmente excel), Libre Office(especialmente Cal), Numbers (MacOs), Mendeley, Latex, Postman, Bruno}.

\section*{Idiomas}
\begin{tabularcv}
\textit{Español}     &	Nativo. \\
\textit{Ingles}      &  Avanzado \textbf{B2}, soy capaz de mantener conversaciones fluidas tanto técnicas como personales .\link {https://drive.google.com/open?id=1vMRxMQEZUpZY7qGLgneV0Vp1bBCL4sXO}\\
\end{tabularcv}

\section*{Intereses}
Practicar deportes, socializar, hacer cosas creativas con codigo (sonic pi, algo contests), aficionado de los juegos (videojuegos, de mesa, cartas...), aprendedor "pasivo", en tiempos muertos me gusta ver que cosas nuevas puedo aprender, para la vida en general.

\section*{Referencias}
Disponibles, en caso de contacto.
    
\end{document}

