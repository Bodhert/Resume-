% !TeX program = xelatex
% Run with XeLaTeX

\documentclass[
    changecolor={111, 156, 45}, 
%    titlecolor=second,
%    colorharmony={wheel,1,3}, 
]{cv-roald}
% Available options for cv-roald documentclass: 
% 
% - changecolor = {R, G, B} (default: {169, 53, 40} = sharelatex lion red)
%   Changes the color of the header bar and the titles. 
%   It expects a RGB color code where RGB go from 0-255.
%
% - colorharmony = {wheel,i,j} (defualt: {wheel,1,3})
%   Choose the color harmony model you want to use.
%     wheel = wheel or twheel
%     i		= number < j
%	  j		= for j-color harmony 
%	    	  j=3 -> three color harmony or color triad
% 	See page 37 of the xcolor manual for more information on commands: 
% 	http://texdoc.net/texmf-dist/doc/latex/xcolor/xcolor.pdf  
% 	Or check http://paletton.com to see how color harmonies work.
%
% - titlecolor = {colorname} (default: main)
%	Change the colors of the titles. You can try: second, black!50, 
%	green!40!yellow, etc.     	
%
%  The colors defined through changecolor and colorharmony are called 
%  *main* and *second* and can be used in the main document.	
\newcommand{\RomanNumeralCaps}[1]
    {\MakeUppercase{\romannumeral #1}}

\begin{document}
\pagestyle{empty} %to remove the page numbers

% This is the header on the first page. It contains your name and contact
% details. 
% \sep inserts a | between items. 
% You can use FontAwesome icons and use \FAspace after a font awesome icon to
% insert a predefined horizontal space after a font awesome icon icon.
\header{Alejandro}{Cordoba}
 {%
  \faMapMarker \FAspace Medellín \sep Colombia \sep%
  \faMobile \FAspace +57 301 323 8497\sep% 
  \faGithub Bodhert \FAspace \link{https://github.com/bodhert}
 }
 {%
  \faEnvelope \FAspace alejandro.cordoba.bodhert@gmail.com \sep% 
  \faLinkedinSquare  \FAspace Alejandro Cordoba Bodhert \FAspace \link{https://bo.linkedin.com/in/alejandro-cordoba-bodhert-746598123}
 }


% Add a picture to the top right of the page. Comment or delete if you do not 
% want a picture
\photo{fotos/foto_hoja_de_vida.jpeg}

\textit{"La disciplina tarde que temprano vencerá la inteligencia" , \textit{Yokoi Kenji}.}

\section{Perfil y Fortalezas}

 Amante de la tecnología y de sacar el mejor provecho de la misma, entendiendo los procesos y otras implicaciones que esta conlleva. Me adapto fácil a entornos que están en constante cambio, poseo buenos conocimientos técnicos y también habilidades \textbf{comunicativas} y de \textbf{liderazgo} , llevando un ambiente de \textbf{respeto} ante todo. Disciplinado, motivado por lo que me apasiona.

\section*{Educación}
% Use tabularcv environment to make a two column environment. The left column 
% is for the dates, the right one is for details of your education for example. 
% You can use the command \worktitle{Study name/Job title}{Location}.
\begin{tabularcv}

2015-Presente   &   \worktitle{Ingeniería de sistemas}{Universidad EAFIT}
                \newline Pasión por la  \textbf{programación}, \textbf{la algoritmia}, \textbf{automatización} de procesos y la administración de sistemas basados en \textbf{linux}, Además poseer conocimiento en diferentes metodologías de desarrollo, tanto tradicionales como ágiles,  con fortaleza en \textbf{scrum}, consciente de las buenas practicas y patrones de desarrollo como \textbf{DevOps} y \textbf{código limpio}.
                \begin{itemize}
				  \item Realización de diversos proyectos que cubren tópicos como programación para sistemas operativos \textbf{(POSIX)}, \textbf{arduino}, \textbf{programación móvil(Ionic,Android)}, \textbf{web services(AWS)}, \textbf{programación funcional}, \textbf{data science}.\link{https://drive.google.com/open?id=1DdJ-zSWe93dtckBD8_2KfLKlCP9mtG4L}, entre otras.
				  \item Integrante del semillero de programación competitiva desde el 2016.
				  \item  Habilidad en la solución de problemas por medio de la \textbf{algoritmia}.
				  \item \textbf{Autodidacta}.
                \end{itemize}
                \\
                
2003-2014   &   \worktitle{Bachiller}{Colegio San José de la Salle}  \link{https://drive.google.com/open?id=1Z0-YVHVlFtDL2GwGxhAoMrZUSDcweuF7}
                \newline Lasallista, Educado en un ambiente que promueve una formación integral en todos los aspectos humanos, promoviendo una sana convivencia. Destaque mi \textbf{liderazgo} llegando a ser representante de los estudiantes y jefe de grupo, en mí ultimo año. \link{https://drive.google.com/open?id=160q1YruZjghGWJOQEOUo_zYKsZ1OVReW}
                \begin{itemize}
				  \item Realización de un proyecto \textbf{investigativo} enfocado al comportamiento social de los estudiantes y como afecta esto su desempeño académico.
                  \item \textbf{Orador} en \RomanNumeralCaps{2} foro de ciencias sociales.
                \end{itemize} 

                
\end{tabularcv}

\section*{Experiencia}
\begin{tabularcv}
2017-presente   &   \worktitle{Monitor investigativo}{Universidad EAFIT (Departamento de ingeniería de producción)}
                \newline Aplicación para el consumo de energía en carros eléctricos.
                \begin{itemize}
                % Example use of *maincolor* and *secondcolor* in the main document.    
                  \item Programación en \textbf{Java} de servicios backend \textbf{(Spring)}, con herramientas free source como \textbf{osrm} y \textbf{postgres}.
                  \item Administración de servidores \textbf{Debian}.
                  \item Creacion y uso de servicios \textbf{REST}.
                  \item Creación de herramientas en java para facilitar y agilizar los quehaceres investigativos.
				  \item Uso de \textbf{linux shell} para configuraciones y administracion del sistema. 
                \end{itemize} 

                
\end{tabularcv}   

\section*{Logros}
% Use \link{URL} to place an link. It inserts a fontawsome logo, where the color is determined from the color harmony settings.
\begin{tabularcv}	
2017    &   Mención honorable en maraton Sur Americana (región norte) ACM ICPC. \link{https://drive.google.com/open?id=1GoiMosLKJv-FpBzl04NuQETugrnbYnaB}\\
2017	&	Puesto 32 en maratón Nacional de programación ACM ICPC.  \link{https://drive.google.com/open?id=1dNH9uwUQMZJxrwjlYxjF_f4irCCxR1Gn}\\
2015    &   Puesto 41 en maratón Nacional de programación ACM ICPC  . \link{https://drive.google.com/file/d/1Zkp_3n9bve-QKLGidpJzQSrKxTxuB_bi/view?usp=sharing} \\

                
\end{tabularcv}

\section*{Habilidades de software}
	\textit{Intermedio:}  \textbf{C++}, \textbf{Java}.\\
	\textit{Basico:} \textbf{Javascript, Python, Postgres, MongoDB, Ruby, C\#, Visual Basic, Scala,}  \textbf{JUnit},\textbf{Jenkins}.\\
	\textit{Sistemas operativos:} \textbf{Linux, Windows.}\\
	\textit{Usos de:} \textbf{linux shell} , \textbf{git} y \textbf{subversion}.\\
	\textit{Frameworks:} \textbf{Spring, node, rails, Ionic, Android Studio}. \\
	\textit{Virtualizacion:} \textbf{Docker, Virtual Box}.\\
	\textit{Hardware:} \textbf{arduino, raspberry pi}.\\
	\textit{Otras herramientas:} \textbf{Microsoft office, Libre Office, mendeley, Latex}.

\section*{Idiomas}
\begin{tabularcv}
\textit{Español}     &	Nativo. \\
\textit{Ingles}      &  Intermedio \textbf{B2}, con capacidad para mantener conversaciones fluidas tanto técnicas como personales .\link {https://drive.google.com/open?id=1vMRxMQEZUpZY7qGLgneV0Vp1bBCL4sXO} ,\\
\end{tabularcv}

\section*{Intereses}
Practicar deportes, socializar, solución de problemas computacionales por diversión, videojuegos que implique pensamiento analítico y estratégico.

\section*{Referencias}
Disponibles, en caso de contactarme.
    
\end{document}

